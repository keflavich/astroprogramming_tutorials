%\documentclass[11pt,letterpaper,notitlepage,onesided]{tex/nwh_hw}
%\documentclass[11pt,letterpaper,notitlepage]{article}
\documentclass{article}
% Change "article" to "report" to get rid of page number on title page
\usepackage{amsmath,amsfonts,amsthm,amssymb}
\usepackage{setspace}
\usepackage{textcomp}
\usepackage{listings}
\lstset{basicstyle=\ttfamily} % <<< This line added
\lstset{upquote=true}
\lstset{breakatwhitespace=true}
\renewcommand{\ttdefault}{cmtt}

\usepackage{Tabbing}
\usepackage{textcomp}
\usepackage{fancyhdr}
\usepackage{lastpage}
\usepackage{extramarks}
\usepackage{chngpage}
\usepackage{soul,color}
\usepackage{graphicx,float,wrapfig}
\usepackage{parskip}
\usepackage[utf8]{inputenc}
\usepackage[T1]{fontenc}

% In case you need to adjust margins:
\topmargin=-0.45in      %
\evensidemargin=0in     %
\oddsidemargin=0in      %
\textwidth=6.5in        %
\textheight=9.0in       %
\headsep=0.25in         %

% Homework Specific Information
\newcommand{\hmwkTitle}{Tutorial: Arrays}
\newcommand{\hmwkDueDate}{September 4th}
\newcommand{\hmwkClass}{ASTR 2600}
\newcommand{\hmwkClassTime}{4:00 PM T/Th}
\newcommand{\hmwkClassInstructor}{Adam Ginsburg}
\newcommand{\hmwkAuthorName}{Dewey Anderson}

% Setup the header and footer
\pagestyle{fancy}                                                       %
%\lhead{\hmwkAuthorName}                                                 %
\chead{\hmwkClass\: \hmwkTitle}  %
\rhead{\firstxmark}                                                     %
\lfoot{\lastxmark}                                                      %
\cfoot{}                                                                %
\rfoot{Page\ \thepage\ of\ \pageref{LastPage}}                          %
\renewcommand\headrulewidth{0.4pt}                                      %
\renewcommand\footrulewidth{0.4pt}                                      %

\usepackage[utf8]{inputenc}
\usepackage[unicode=true]{hyperref}
\hypersetup{breaklinks=true,
            bookmarks=true,
            pdfauthor={},
            pdftitle={Connecting to the cosmos computer from home using Microsoft Windows},
            colorlinks=true,
            urlcolor=blue,
            linkcolor=magenta,
            pdfborder={0 0 0}}

% This is used to trace down (pin point) problems
% in latexing a document:
%\tracingall

%%%%%%%%%%%%%%%%%%%%%%%%%%%%%%%%%%%%%%%%%%%%%%%%%%%%%%%%%%%%%
% Some tools
\newcommand{\enterProblemHeader}[1]{\nobreak\extramarks{#1}{#1 continued on next page\ldots}\nobreak%
                                    \nobreak\extramarks{#1 (continued)}{#1 continued on next page\ldots}\nobreak}%
\newcommand{\exitProblemHeader}[1]{\nobreak\extramarks{#1 (continued)}{#1 continued on next page\ldots}\nobreak%
                                   \nobreak\extramarks{#1}{}\nobreak}%

\newlength{\labelLength}
\newcommand{\labelAnswer}[2]
  {\settowidth{\labelLength}{#1}%
   \addtolength{\labelLength}{0.25in}%
   \changetext{}{-\labelLength}{}{}{}%
   \noindent\fbox{\begin{minipage}[c]{\columnwidth}#2\end{minipage}}%
   \marginpar{\fbox{#1}}%

   % We put the blank space above in order to make sure this
   % \marginpar gets correctly placed.
   \changetext{}{+\labelLength}{}{}{}}%

\setcounter{secnumdepth}{0}
\newcommand{\homeworkProblemName}{}%
\newcounter{homeworkProblemCounter}%
\newenvironment{homeworkProblem}[1][Problem \arabic{homeworkProblemCounter}]%
  {\stepcounter{homeworkProblemCounter}%
   \renewcommand{\homeworkProblemName}{#1}%
   \section{\homeworkProblemName}%
   \enterProblemHeader{\homeworkProblemName}}%
  {\exitProblemHeader{\homeworkProblemName}}%

\newcommand{\problemAnswer}[1]
  {\noindent\fbox{\begin{minipage}[c]{\columnwidth}#1\end{minipage}}}%

\newcommand{\problemLAnswer}[1]
  {\labelAnswer{\homeworkProblemName}{#1}}

\newcommand{\homeworkSectionName}{}%
\newlength{\homeworkSectionLabelLength}{}%
\newenvironment{homeworkSection}[1]%
  {% We put this space here to make sure we're not connected to the above.
   % Otherwise the changetext can do funny things to the other margin

   \renewcommand{\homeworkSectionName}{#1}%
   \settowidth{\homeworkSectionLabelLength}{\homeworkSectionName}%
   \addtolength{\homeworkSectionLabelLength}{0.25in}%
   \changetext{}{-\homeworkSectionLabelLength}{}{}{}%
   \subsection{\homeworkSectionName}%
   \enterProblemHeader{\homeworkProblemName\ [\homeworkSectionName]}}%
  {\enterProblemHeader{\homeworkProblemName}%

   % We put the blank space above in order to make sure this margin
   % change doesn't happen too soon (otherwise \sectionAnswer's can
   % get ugly about their \marginpar placement.
   \changetext{}{+\homeworkSectionLabelLength}{}{}{}}%

\newcommand{\sectionAnswer}[1]
  {% We put this space here to make sure we're disconnected from the previous
   % passage

   \noindent\fbox{\begin{minipage}[c]{\columnwidth}#1\end{minipage}}%
   \enterProblemHeader{\homeworkProblemName}\exitProblemHeader{\homeworkProblemName}%
   \marginpar{\fbox{\homeworkSectionName}}%

   % We put the blank space above in order to make sure this
   % \marginpar gets correctly placed.
   }%

%%%%%%%%%%%%%%%%%%%%%%%%%%%%%%%%%%%%%%%%%%%%%%%%%%%%%%%%%%%%%


%%%%%%%%%%%%%%%%%%%%%%%%%%%%%%%%%%%%%%%%%%%%%%%%%%%%%%%%%%%%%
% Make title
\title{\vspace{2in}\textmd{\textbf{\hmwkClass:\ \hmwkTitle}}\\\normalsize\vspace{0.1in}\small{Due\ on\ \hmwkDueDate}\\\vspace{0.1in}\large{}\vspace{3in}}
\date{}
%\author{\textbf{\hmwkAuthorName}}
%%%%%%%%%%%%%%%%%%%%%%%%%%%%%%%%%%%%%%%%%%%%%%%%%%%%%%%%%%%%%

\begin{document}
\begin{spacing}{1.0}
%\maketitle
\newpage


\section{String Manipulation}

We want to extract information from strings.
Create the following variables (if you want, you can copy and paste them from
\verb|/home/shared/astr2600/data/tutorial3_data.txt| - use \verb|cat| to see
its contents):

\begin{lstlisting}
R = "Rigel      magB=  0.9 magV= 0.12"
S = "Sirius     magB=-1.46 magV=-1.46"
V = "Vega       magB= 0.03 magV= 0.03"
B = "Betelgeuse magB= 2.27 magV= 0.42"
\end{lstlisting}

Our goal is to extract B and V magnitudes for these stars and enter them into
arrays.

First, find out where the magB value is stored.  To do this, let's look for the string
\verb|"magB="|.

\begin{lstlisting}
magBstart = strpos(R,"magB=")
print,magBstart
\end{lstlisting}

If you've done this correctly, magBstart should be 11.  OK, great, but we don't care where
the string \verb|"magB"| starts, we care where it \emph{ends}.  So, let's use \verb|strlen|:

\begin{lstlisting}
magBlen = strlen("magB=")
magBend = magBstart + magBlen
\end{lstlisting}

All good so far.  Let's do the same for magV:

\begin{lstlisting}
magVstart = strpos(R,"magV=")
magVlen = strlen("magV=")
magVend = magVstart + magVlen
\end{lstlisting}

Now we actually want to extract the magnitudes.  We'll use \verb|strmid| for that purpose.

\begin{lstlisting}
magB = strmid(R, magBend, magBlen)
magV = strmid(R, magVend, magVlen)
print,"Rigel magB is ",magB," and magV is ",magV
\end{lstlisting}

Great, we've read one of the strings!  This whole process was pretty tedious, so we ideally
don't want to have to re-do it.  We can note that the strings were all of the same length:

\begin{lstlisting}
print,strlen(B)
print,strlen(V)
print,strlen(S)
print,strlen(R)
\end{lstlisting}

and that they all have magB and magV in the same places.  So all we need to do
is the \verb|strmid| step, not the others.

Since we're making arrays, we'll do this all as one big step\dots
\begin{lstlisting}[breaklines]
magB = [strmid(R, magBend, magBlen),strmid(B, magBend, magBlen),strmid(S, magBend, magBlen),strmid(V, magBend, magBlen)]
magV = [strmid(R, magVend, magVlen),strmid(B, magVend, magVlen),strmid(S, magVend, magVlen),strmid(V, magVend, magVlen)]
print,magB
print,magV
help,magB,magV
\end{lstlisting}

But these are still of type string!  Let's make them floats:

\begin{lstlisting}
magB = float(magB)
magV = float(magV)
\end{lstlisting}

Now, finally, we have some arrays with real data in them.  We'll make a
color-magnitude diagram using IDL's plot routine.  Try each of these commands in order,
they'll look slightly different:

\begin{lstlisting}
; this first one will look a bit strange
plot,magB-magV,magV
; this is a little better
plot,magB-magV,magV,psym=1
; maybe stars are best
plot,magB-magV,magV,psym=2
; Make it colorful
plot,magB-magV,magV,psym=2
oplot,magB-magV,magV,psym=2,color='0000FF'x,symsize=10
\end{lstlisting}

OK, those plots didn't look great - not many data points.  Let's load more data and try again.
As above, copy \& paste lines from \verb|/home/shared/astr2600/data/tutorial3_data.txt|:

\begin{lstlisting}[breaklines]
pleiadesB = [2.806, 3.54, 3.612, 3.812, 4.113, 4.199, 4.967, 5.406, 5.585, 5.727, 6.124, 6.30, 6.415, 6.586, 6.87, 6.85]
pleiadesV = [2.873, 3.62, 3.705, 3.871, 4.164, 4.291, 5.048, 5.448, 5.651, 5.761, 6.172, 6.28, 6.430, 6.606, 6.81, 6.83]
\end{lstlisting}

Then, let's try the same plots as above:
\begin{lstlisting}
; this first one will look a bit strange
plot,pleiadesB-pleiadesV,pleiadesV
; this is a little better
plot,pleiadesB-pleiadesV,pleiadesV,psym=1
; maybe stars are best
plot,pleiadesB-pleiadesV,pleiadesV,psym=2
; Make it colorful
; note the single quotes (not double) in this line
oplot,pleiadesB-pleiadesV,pleiadesV,psym=2,color='0000FF'x,symsize=3
\end{lstlisting}

To give you some idea of the astrophysics involved, plus get you used to labeling things,
we'll add a title and axis labels

\begin{lstlisting}[breaklines]
plot,pleiadesB-pleiadesV,pleiadesV,psym=2,xtitle='B-V',ytitle='V magnitude',title='Pleiades Color-Magnitude Diagram'
oplot,pleiadesB-pleiadesV,pleiadesV,psym=2,color='0000FF'x,symsize=3
\end{lstlisting}

Finally, for a real color-magnitude diagram, we want the brightest things plotted at the top.
\begin{lstlisting}[breaklines,upquote=true]
plot,pleiadesB-pleiadesV,pleiadesV,psym=2,xtitle='B-V',ytitle='V magnitude',title='Pleiades Color-Magnitude Diagram',yrange=[8,2]
oplot,pleiadesB-pleiadesV,pleiadesV,psym=2,color='0000FF'x,symsize=3
\end{lstlisting}

Color-magnitude diagrams are useful plots for determining properties of star
clusters and galaxies.  The y-axis, the brightness of the source, is strongly
dependent on distance and the star's luminosity, while the x-axis depends on
the star's intrinsic properties, but not distance.  When looking at a cluster
or a galaxy, all stars are at (approximately) the same distance, and therefore
the y-axis only depends on the luminosity.  The Pleiades cluster is one of the
nearest and most massive clusters we have complete information on (meaning we
think we know its distance very accurately), so it serves as a calibration
target for stellar properties.

\end{spacing}
\end{document}

