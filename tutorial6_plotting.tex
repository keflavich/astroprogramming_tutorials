%\documentclass[11pt,letterpaper,notitlepage,onesided]{tex/nwh_hw}
%\documentclass[11pt,letterpaper,notitlepage]{article}
\documentclass{article}
% Change "article" to "report" to get rid of page number on title page
\usepackage{amsmath,amsfonts,amsthm,amssymb}
\usepackage{setspace}
\usepackage{textcomp}
\usepackage{listings}
\lstset{basicstyle=\ttfamily} % <<< This line added
\lstset{upquote=true}
\lstset{breakatwhitespace=true}
\renewcommand{\ttdefault}{cmtt}

\usepackage{Tabbing}
\usepackage{textcomp}
\usepackage{fancyhdr}
\usepackage{lastpage}
\usepackage{extramarks}
\usepackage{chngpage}
\usepackage{soul,color}
\usepackage{graphicx,float,wrapfig}
\usepackage{parskip}
\usepackage[utf8]{inputenc}
\usepackage[T1]{fontenc}

% In case you need to adjust margins:
\topmargin=-0.45in      %
\evensidemargin=0in     %
\oddsidemargin=0in      %
\textwidth=6.5in        %
\textheight=9.0in       %
\headsep=0.25in         %

% Homework Specific Information
\newcommand{\hmwkTitle}{Tutorial: Plotting 1}
\newcommand{\hmwkDueDate}{DATE, 4:00 PM}
\newcommand{\hmwkClass}{ASTR 2600}
\newcommand{\hmwkClassTime}{4:00 PM T/Th}
\newcommand{\hmwkClassInstructor}{Adam Ginsburg}
\newcommand{\hmwkAuthorName}{Dewey Anderson}

% Setup the header and footer
\pagestyle{fancy}                                                       %
%\lhead{\hmwkAuthorName}                                                 %
\chead{\hmwkClass\: \hmwkTitle}  %
\rhead{\firstxmark}                                                     %
\lfoot{\lastxmark}                                                      %
\cfoot{}                                                                %
\rfoot{Page\ \thepage\ of\ \pageref{LastPage}}                          %
\renewcommand\headrulewidth{0.4pt}                                      %
\renewcommand\footrulewidth{0.4pt}                                      %

\usepackage[utf8]{inputenc}
\usepackage[unicode=true]{hyperref}
\hypersetup{breaklinks=true,
            bookmarks=true,
            pdfauthor={},
            pdftitle={Connecting to the cosmos computer from home using Microsoft Windows},
            colorlinks=true,
            urlcolor=blue,
            linkcolor=magenta,
            pdfborder={0 0 0}}

% This is used to trace down (pin point) problems
% in latexing a document:
%\tracingall

%%%%%%%%%%%%%%%%%%%%%%%%%%%%%%%%%%%%%%%%%%%%%%%%%%%%%%%%%%%%%
% Some tools
\newcommand{\enterProblemHeader}[1]{\nobreak\extramarks{#1}{#1 continued on next page\ldots}\nobreak%
                                    \nobreak\extramarks{#1 (continued)}{#1 continued on next page\ldots}\nobreak}%
\newcommand{\exitProblemHeader}[1]{\nobreak\extramarks{#1 (continued)}{#1 continued on next page\ldots}\nobreak%
                                   \nobreak\extramarks{#1}{}\nobreak}%

\newlength{\labelLength}
\newcommand{\labelAnswer}[2]
  {\settowidth{\labelLength}{#1}%
   \addtolength{\labelLength}{0.25in}%
   \changetext{}{-\labelLength}{}{}{}%
   \noindent\fbox{\begin{minipage}[c]{\columnwidth}#2\end{minipage}}%
   \marginpar{\fbox{#1}}%

   % We put the blank space above in order to make sure this
   % \marginpar gets correctly placed.
   \changetext{}{+\labelLength}{}{}{}}%

\setcounter{secnumdepth}{0}
\newcommand{\homeworkProblemName}{}%
\newcounter{homeworkProblemCounter}%
\newenvironment{homeworkProblem}[1][Problem \arabic{homeworkProblemCounter}]%
  {\stepcounter{homeworkProblemCounter}%
   \renewcommand{\homeworkProblemName}{#1}%
   \section{\homeworkProblemName}%
   \enterProblemHeader{\homeworkProblemName}}%
  {\exitProblemHeader{\homeworkProblemName}}%

\newcommand{\problemAnswer}[1]
  {\noindent\fbox{\begin{minipage}[c]{\columnwidth}#1\end{minipage}}}%

\newcommand{\problemLAnswer}[1]
  {\labelAnswer{\homeworkProblemName}{#1}}

\newcommand{\homeworkSectionName}{}%
\newlength{\homeworkSectionLabelLength}{}%
\newenvironment{homeworkSection}[1]%
  {% We put this space here to make sure we're not connected to the above.
   % Otherwise the changetext can do funny things to the other margin

   \renewcommand{\homeworkSectionName}{#1}%
   \settowidth{\homeworkSectionLabelLength}{\homeworkSectionName}%
   \addtolength{\homeworkSectionLabelLength}{0.25in}%
   \changetext{}{-\homeworkSectionLabelLength}{}{}{}%
   \subsection{\homeworkSectionName}%
   \enterProblemHeader{\homeworkProblemName\ [\homeworkSectionName]}}%
  {\enterProblemHeader{\homeworkProblemName}%

   % We put the blank space above in order to make sure this margin
   % change doesn't happen too soon (otherwise \sectionAnswer's can
   % get ugly about their \marginpar placement.
   \changetext{}{+\homeworkSectionLabelLength}{}{}{}}%

\newcommand{\sectionAnswer}[1]
  {% We put this space here to make sure we're disconnected from the previous
   % passage

   \noindent\fbox{\begin{minipage}[c]{\columnwidth}#1\end{minipage}}%
   \enterProblemHeader{\homeworkProblemName}\exitProblemHeader{\homeworkProblemName}%
   \marginpar{\fbox{\homeworkSectionName}}%

   % We put the blank space above in order to make sure this
   % \marginpar gets correctly placed.
   }%

%%%%%%%%%%%%%%%%%%%%%%%%%%%%%%%%%%%%%%%%%%%%%%%%%%%%%%%%%%%%%


%%%%%%%%%%%%%%%%%%%%%%%%%%%%%%%%%%%%%%%%%%%%%%%%%%%%%%%%%%%%%
% Make title
\title{\vspace{2in}\textmd{\textbf{\hmwkClass:\ \hmwkTitle}}\\\normalsize\vspace{0.1in}\small{Due\ on\ \hmwkDueDate}\\\vspace{0.1in}\large{}\vspace{3in}}
\date{}
%\author{\textbf{\hmwkAuthorName}}
%%%%%%%%%%%%%%%%%%%%%%%%%%%%%%%%%%%%%%%%%%%%%%%%%%%%%%%%%%%%%

\begin{document}
\begin{spacing}{1.0}
%\maketitle
\newpage



\section{Tutorial: Plotting (1)}

We'll plot some things in IDL.  This is a simple intro, but we'll do
more plotting tutorials later on.

Take advantage of copy \& paste and the up arrow in this tutorial.  If you
select text with the mouse, the text is ``copied into the buffer'' (as if you'd
gone to the ``Edit'' menu and pressed ``Copy'' in a word processing program).
You can past it by middle-clicking wherever you want it pasted.

Open IDL.  Start a journal file called
\verb|tutorial6_plotting_[today'sdate].pro|.  Replace the part in square braces
with today's date!

We'll make two \verb|x| arrays: one ``high resolution'' and one ``low resolution''

\begin{lstlisting}
    x1 = findgen(1000)/999*4*!pi
    x2 = findgen(10)/9*4*!pi
\end{lstlisting}

Plot and overplot some functions (remember you can use the up arrow key to recall
previous commands):

\begin{lstlisting}
    plot,x1,sin(x1)
    oplot,x1,sin(x1)^2
    oplot,x2,sin(x2)
    oplot,x2,sin(x2)^2
\end{lstlisting}

That looks hideous.  We'll add some color.

\begin{lstlisting}
    plot,x1,sin(x1)
    oplot,x1,sin(x1)^2,color='0000FF'x
    oplot,x2,sin(x2),color='00FF00'x
    oplot,x2,sin(x2)^2,color='FF0000'x
\end{lstlisting}

Well\dots it's colorful, at least.

We can make the lines thicker:

\begin{lstlisting}
    oplot,x1,sin(x1)^2,color='0000FF'x,thick=3
\end{lstlisting}

Notice that there's some blank space on the right side.  This can be cleaned up with
the \verb|xstyle| keyword:

\begin{lstlisting}[breaklines,upquote=true]
    plot,x1,sin(x1),xstyle=1,thick=3
    oplot,x1,sin(x1)^2,color='0000FF'x,thick=3
    oplot,x2,sin(x2),color='00FF00'x,thick=3
    oplot,x2,sin(x2)^2,color='FF0000'x,thick=3
\end{lstlisting}

What does \verb|xstyle| mean?  Open the online help:

\verb|?plot|

Scroll down until you find the keyword \verb|style|.  Read the help.  Also note
all the myriad options on the way down!

Let's build up the longest sensible plto command we can (it might be a good
idea to write this in a text file and copy and paste it so you can scroll
around the line more easily):

\begin{lstlisting}[breaklines,upquote=true]
    plot, x1, sin(x1), xstyle=1, thick=3, title='My Plot', xtitle='The X Axis', ytitle='The Y Axis', ytickname=['minus 1','minus 1/2','0','1/2','1']
    oplot,x1,sin(x1)^2,color='0000FF'x,thick=3
    oplot,x2,sin(x2),color='00FF00'x,thick=3
    oplot,x2,sin(x2)^2,color='FF0000'x,thick=3
\end{lstlisting}

OK, clearly you can do a lot with a single IDL command, but it becomes
overwhelming and excessive rather quickly.

There's a way around this: the ``line continuation character'' \verb|$|

\begin{lstlisting}[breaklines,upquote=true]
    plot, x1, sin(x1), xstyle=1, thick=3, title='My Plot', xtitle='The X Axis',$
        ytitle='The Y Axis', ytickname=['minus 1','minus 1/2','0','1/2','1']
\end{lstlisting}

Now you've entered two lines of text, but IDL treats them as one.  This is a
way to make your code neater: if you have a very long command, you can break it
up the way you would break up normal text in a paragraph as long as you tell
IDL by adding a \verb|$| at the end.

ASIDE: You must use single quotes, not double quotes, if you have a number as
the first character.  For some quirky reason, IDL treats anything that looks like
\verb|"0231| or \verb|"#...| (i.e., a double quote followed by any number) as an
``octal'' value.

\section{More plotting - now python}
This is a class about programming, not just about IDL.  It is also a class
about programming for astronomy.  If you decide to work with or for
astronomers, odds are you will see BOTH IDL and python being used.  So, I want
you to be able to read and at least do minimal tasks in both.

To start python, run the command

\verb|ipython -pylab|

This is not `vanilla python' - \texttt{ipython} is a special environment
designed for interactive use.  \texttt{pylab} means that the syntax (the names
of functions and the way you call them) is meant to resemble matlab.  It's also
pretty similar to IDL, but there are some key differences.

We'll do the same plots as in IDL:

\begin{lstlisting}
    # IDL was: x1 = findgen(1000)/999*4*!pi
    x1 = linspace(0,4*pi,1000) 
    # x2 = findgen(10)/9*4*!pi
    x2 = linspace(0,4*pi,10) 
    plot(x1,sin(x1))
    plot(x1,sin(x1)**2)
    plot(x2,sin(x2))
    plot(x2,sin(x2)**2)
\end{lstlisting}

Note the differences:
\begin{enumerate}
    \item Python defaults to a white background with some buttons.  
    \item Python has the ``convenience'' command \verb|linspace|
    \item Python does \emph{not} clear the plot by default: clearing the plot
        requires the command \verb|clf()|
    \item Python commands are followed by parentheses, while IDL procedures
        were followed by commas.  We'll revisit this later: IDL uses two
        different syntaxes for ``procedures'' and ``functions''\dots
    \item The exponent in python is \verb|**|, in IDL is \verb|^|
    \item Comments in python start with \verb|#| instead of \verb|;|
\end{enumerate}

\end{spacing}
\end{document}

