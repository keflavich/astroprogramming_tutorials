%\documentclass[11pt,letterpaper,notitlepage,onesided]{tex/nwh_hw}
%\documentclass[11pt,letterpaper,notitlepage]{article}
\documentclass{article}
% Change "article" to "report" to get rid of page number on title page
\usepackage{amsmath,amsfonts,amsthm,amssymb}
\usepackage{setspace}
\usepackage{textcomp}
\usepackage{listings}
\lstset{basicstyle=\ttfamily} % <<< This line added
\lstset{upquote=true}
\lstset{breakatwhitespace=true}
\renewcommand{\ttdefault}{cmtt}

\usepackage[compact]{titlesec}
%\titlespacing{section}{0mm}{.5mm}{.5mm}[]


\usepackage{Tabbing}
\usepackage{textcomp}
\usepackage{fancyhdr}
\usepackage{lastpage}
\usepackage{extramarks}
\usepackage{chngpage}
\usepackage{soul,color}
\usepackage{graphicx,float,wrapfig}
\usepackage{parskip}
\usepackage[utf8]{inputenc}
\usepackage[T1]{fontenc}

% In case you need to adjust margins:
\topmargin=-0.45in      %
\evensidemargin=0in     %
\oddsidemargin=0in      %
\textwidth=6.5in        %
\textheight=9.0in       %
\headsep=0.25in         %

% Homework Specific Information
\newcommand{\hmwkTitle}{Tutorial: Loops}
\newcommand{\hmwkDueDate}{DATE, 4:00 PM}
\newcommand{\hmwkClass}{ASTR 2600}
\newcommand{\hmwkClassTime}{4:00 PM T/Th}
\newcommand{\hmwkClassInstructor}{Adam Ginsburg}
\newcommand{\hmwkAuthorName}{Dewey Anderson}

% Setup the header and footer
\pagestyle{fancy}                                                       %
%\lhead{\hmwkAuthorName}                                                 %
\chead{\hmwkClass\: \hmwkTitle}  %
\rhead{\firstxmark}                                                     %
\lfoot{\lastxmark}                                                      %
\cfoot{}                                                                %
\rfoot{Page\ \thepage\ of\ \pageref{LastPage}}                          %
\renewcommand\headrulewidth{0.4pt}                                      %
\renewcommand\footrulewidth{0.4pt}                                      %

\usepackage[utf8]{inputenc}
\usepackage[unicode=true]{hyperref}
\hypersetup{breaklinks=true,
            bookmarks=true,
            pdfauthor={},
            pdftitle={Connecting to the cosmos computer from home using Microsoft Windows},
            colorlinks=true,
            urlcolor=blue,
            linkcolor=magenta,
            pdfborder={0 0 0}}

% This is used to trace down (pin point) problems
% in latexing a document:
%\tracingall

%%%%%%%%%%%%%%%%%%%%%%%%%%%%%%%%%%%%%%%%%%%%%%%%%%%%%%%%%%%%%
% Some tools
\newcommand{\enterProblemHeader}[1]{\nobreak\extramarks{#1}{#1 continued on next page\ldots}\nobreak%
                                    \nobreak\extramarks{#1 (continued)}{#1 continued on next page\ldots}\nobreak}%
\newcommand{\exitProblemHeader}[1]{\nobreak\extramarks{#1 (continued)}{#1 continued on next page\ldots}\nobreak%
                                   \nobreak\extramarks{#1}{}\nobreak}%

\newlength{\labelLength}
\newcommand{\labelAnswer}[2]
  {\settowidth{\labelLength}{#1}%
   \addtolength{\labelLength}{0.25in}%
   \changetext{}{-\labelLength}{}{}{}%
   \noindent\fbox{\begin{minipage}[c]{\columnwidth}#2\end{minipage}}%
   \marginpar{\fbox{#1}}%

   % We put the blank space above in order to make sure this
   % \marginpar gets correctly placed.
   \changetext{}{+\labelLength}{}{}{}}%

\setcounter{secnumdepth}{0}
\newcommand{\homeworkProblemName}{}%
\newcounter{homeworkProblemCounter}%
\newenvironment{homeworkProblem}[1][Problem \arabic{homeworkProblemCounter}]%
  {\stepcounter{homeworkProblemCounter}%
   \renewcommand{\homeworkProblemName}{#1}%
   \section{\homeworkProblemName}%
   \enterProblemHeader{\homeworkProblemName}}%
  {\exitProblemHeader{\homeworkProblemName}}%

\newcommand{\problemAnswer}[1]
  {\noindent\fbox{\begin{minipage}[c]{\columnwidth}#1\end{minipage}}}%

\newcommand{\problemLAnswer}[1]
  {\labelAnswer{\homeworkProblemName}{#1}}

\newcommand{\homeworkSectionName}{}%
\newlength{\homeworkSectionLabelLength}{}%
\newenvironment{homeworkSection}[1]%
  {% We put this space here to make sure we're not connected to the above.
   % Otherwise the changetext can do funny things to the other margin

   \renewcommand{\homeworkSectionName}{#1}%
   \settowidth{\homeworkSectionLabelLength}{\homeworkSectionName}%
   \addtolength{\homeworkSectionLabelLength}{0.25in}%
   \changetext{}{-\homeworkSectionLabelLength}{}{}{}%
   \subsection{\homeworkSectionName}%
   \enterProblemHeader{\homeworkProblemName\ [\homeworkSectionName]}}%
  {\enterProblemHeader{\homeworkProblemName}%

   % We put the blank space above in order to make sure this margin
   % change doesn't happen too soon (otherwise \sectionAnswer's can
   % get ugly about their \marginpar placement.
   \changetext{}{+\homeworkSectionLabelLength}{}{}{}}%

\newcommand{\sectionAnswer}[1]
  {% We put this space here to make sure we're disconnected from the previous
   % passage

   \noindent\fbox{\begin{minipage}[c]{\columnwidth}#1\end{minipage}}%
   \enterProblemHeader{\homeworkProblemName}\exitProblemHeader{\homeworkProblemName}%
   \marginpar{\fbox{\homeworkSectionName}}%

   % We put the blank space above in order to make sure this
   % \marginpar gets correctly placed.
   }%

%%%%%%%%%%%%%%%%%%%%%%%%%%%%%%%%%%%%%%%%%%%%%%%%%%%%%%%%%%%%%


%%%%%%%%%%%%%%%%%%%%%%%%%%%%%%%%%%%%%%%%%%%%%%%%%%%%%%%%%%%%%
% Make title
\title{\vspace{2in}\textmd{\textbf{\hmwkClass:\ \hmwkTitle}}\\\normalsize\vspace{0.1in}\small{Due\ on\ \hmwkDueDate}\\\vspace{0.1in}\large{}\vspace{3in}}
\date{}
%\author{\textbf{\hmwkAuthorName}}
%%%%%%%%%%%%%%%%%%%%%%%%%%%%%%%%%%%%%%%%%%%%%%%%%%%%%%%%%%%%%

\begin{document}
\begin{spacing}{1.0}
%\maketitle
\newpage



\section{Tutorial: Loops \& Flow Control}

This tutorial will involve exercises to get you accustomed to creating and
manipulating flow control constructs.  It will move on to complex uses of
different data structures and comparison operators.

Create a \emph{program} called \verb|isthenumberbig.pro|.  As you might infer
from the name, it will tell you if a number is big.  To create a program, open
a file with the name as specified above (\verb|isthenumberbig.pro|) in gedit.
Please do this in \emph{your} ASTR2600 directory that you have done 
\verb|git init| in, \emph{not} your ASTR2600-fork directory.  If you'd like,
you can create a \verb|tutorial| subdirectory and open the file in there.  You
probably should.

This program shall do the following:
\begin{enumerate}
    \item Read in a number from the user
    \item Say ``That is a big number'' if the number is greater than 100
    \item Repeat unless the user has input the number -1
\end{enumerate}

This program should include at least one loop construct; it is up to you to
figure out which kind of loop (\verb|while|, \verb|repeat until|, \verb|for|)
is most appropriate.  It should include an \verb|if| statement.  It may require
multiple \verb|if| statements and it may require the \verb|break| statement.
It may also be useful to have a \verb|case| statement.  The \verb|read|
statement should have a descriptive prompt.  And, as always, there should be
comments throughout!

Make sure your code works (test it with a few numbers: -10, 50, 150, 1d30).

\section{\texttt{stop}}
Now insert a \verb|stop| statement in your code, somewhere inside the loop.
Run the code.  You will be dumped out with the following prompt:
\begin{lstlisting}
% Compiled module: $MAIN$.
% Stop encountered: $MAIN$             ##
\end{lstlisting}   
where \verb|##| is the line number of the \verb|stop| statement.

At this point, examine your variables: Try using the \verb|print| and
\verb|help| statements.  See what's going on.

When you're done, use \verb|.continue| to continue.  You'll be stopped again,
though, since you're inside a loop.  Use \verb|help| and \verb|print| again to
see what's changed.  Then \verb|.continue| again, then quit.  Once you're done
examining the inside of your loop, remove the \verb|stop| statement.

\begin{samepage}
\section{crash!}
Now, in the same location where you had the \verb|stop| statement, add the
following line:\\
\verb|y = zzzz^2|

This line will cause a crash, with the text: \\
\begin{lstlisting}
% Variable is undefined: ZZZZ.
% Execution halted at: $MAIN$             15
\end{lstlisting}

Note that the ``crash'' has nearly the same effect as the stop statement:
you are back in interactive mode, and need to use \verb|.c| to continue.  As
long as the variable \verb|zzzz| does not exist, you'll continue to be dropped
out just like with \verb|stop| above.  

However, try defining the number \verb|zzzz=5| (or any number of your choosing;
it really doesn't matter).  Then do \verb|.c|.  What happens next?
\end{samepage}

\section{Comparison from Text}
Write a function called \verb|compare_values| in a file
\verb|compare_values.pro| that takes 3 inputs: \verb|value1|,
\verb|comparison_name|, and \verb|value2|.  The program should use a
\verb|case| statement that will return the appropriate comparison.  i.e., if
you call \verb|print,compare_values(2,'gt',1)|, the function should return the
value of \verb|2 gt 1|, which is “True" or \verb|1|.

Make sure it can deal with the following comparison operators:\\
\verb|gt lt ge le eq ne < >|

\section{Conditionals}
Create a new file called \verb|test_conditionals.pro|.  This is a reference
file for your own use now and in the future.

You will create a series of ``test conditionals'' in this file by using the
ternary operator.  The tests will take the following form:\\
\verb|print,"Is x gt y?  ",(x gt y) ? "yes" : "no"|

Create tests for the following, but \emph{fill these out by hand}
before you run them! (i.e., circle ``yes'' or ``no'')\\

HINT: To make this easier, you can make use of the \verb|list| and/or the
\verb|hash| construct.  You should use \verb|foreach| loops.  You should
definitely make use of the \verb|compare_values| function you created above.

{\large
\begin{tabular}{|c|c|c|c|}
    \hline
    x & comparison & y & yes or no \\
    \hline
    1 & gt & 0 & yes no \\ 
    1 & > & 0 & yes no \\ 
    5 & > & 0 & yes no \\ 
    4 & > & 0 & yes no \\ 
    1 & > & 7 & yes no \\ 
    5 & > & 7 & yes no \\ 
    4 & > & 7 & yes no \\ 
    `0' & eq & 0 & yes no \\ 
    `1' & eq & 0 & yes no \\ 
    0.0 & eq & 0 & yes no \\ 
    1.0 & eq & 1 & yes no \\ 
    0.5d & eq & 0.5 & yes no \\ 
    0.2d & eq & 0.2 & yes no \\ 
    \verb|2^(-30d)| & eq & \verb|2^(-30e)|& yes no \\ 
    \verb|1.1+2^(-30d)| & eq & \verb|1.1+2^(-30e)|& yes no \\ 
    `yes' & eq & "yes" & yes no \\ 
    \verb|sin(!pi)| & lt & \verb|sin(!dpi)| & yes no \\ 
    \verb|cos(!pi)| & eq & \verb|cos(!dpi)| & yes no \\ 
    \hline
\end{tabular}
}

Add any tests you can think of to this. 

Once you have completed this exercise, \verb|git add| the file, 
\verb|git commit -a| to commit the changes, and \verb|git push| them to your
repository.

\end{spacing}
\end{document}

