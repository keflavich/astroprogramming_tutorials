%\documentclass[11pt,letterpaper,notitlepage,onesided]{tex/nwh_hw}
%\documentclass[11pt,letterpaper,notitlepage]{article}
\documentclass{article}
% Change "article" to "report" to get rid of page number on title page
\usepackage{amsmath,amsfonts,amsthm,amssymb}
\usepackage{setspace}
\usepackage{textcomp}
\usepackage{listings}
\lstset{basicstyle=\ttfamily} % <<< This line added
\lstset{upquote=true}
\lstset{breakatwhitespace=true}
\renewcommand{\ttdefault}{cmtt}

\usepackage{Tabbing}
\usepackage{textcomp}
\usepackage{fancyhdr}
\usepackage{lastpage}
\usepackage{extramarks}
\usepackage{chngpage}
\usepackage{soul,color}
\usepackage{graphicx,float,wrapfig}
\usepackage{parskip}
\usepackage[utf8]{inputenc}
\usepackage[T1]{fontenc}

% In case you need to adjust margins:
\topmargin=-0.45in      %
\evensidemargin=0in     %
\oddsidemargin=0in      %
\textwidth=6.5in        %
\textheight=9.0in       %
\headsep=0.25in         %

% Homework Specific Information
\newcommand{\hmwkTitle}{Tutorial: Forking}
\newcommand{\hmwkDueDate}{DATE, 4:00 PM}
\newcommand{\hmwkClass}{ASTR 2600}
\newcommand{\hmwkClassTime}{4:00 PM T/Th}
\newcommand{\hmwkClassInstructor}{Adam Ginsburg}
\newcommand{\hmwkAuthorName}{Dewey Anderson}

% Setup the header and footer
\pagestyle{fancy}                                                       %
%\lhead{\hmwkAuthorName}                                                 %
\chead{\hmwkClass\: \hmwkTitle}  %
\rhead{\firstxmark}                                                     %
\lfoot{\lastxmark}                                                      %
\cfoot{}                                                                %
\rfoot{Page\ \thepage\ of\ \pageref{LastPage}}                          %
\renewcommand\headrulewidth{0.4pt}                                      %
\renewcommand\footrulewidth{0.4pt}                                      %

\usepackage[utf8]{inputenc}
\usepackage[unicode=true]{hyperref}
\hypersetup{breaklinks=true,
            bookmarks=true,
            pdfauthor={},
            pdftitle={Connecting to the cosmos computer from home using Microsoft Windows},
            colorlinks=true,
            urlcolor=blue,
            linkcolor=magenta,
            pdfborder={0 0 0}}

% This is used to trace down (pin point) problems
% in latexing a document:
%\tracingall

%%%%%%%%%%%%%%%%%%%%%%%%%%%%%%%%%%%%%%%%%%%%%%%%%%%%%%%%%%%%%
% Some tools
\newcommand{\enterProblemHeader}[1]{\nobreak\extramarks{#1}{#1 continued on next page\ldots}\nobreak%
                                    \nobreak\extramarks{#1 (continued)}{#1 continued on next page\ldots}\nobreak}%
\newcommand{\exitProblemHeader}[1]{\nobreak\extramarks{#1 (continued)}{#1 continued on next page\ldots}\nobreak%
                                   \nobreak\extramarks{#1}{}\nobreak}%

\newlength{\labelLength}
\newcommand{\labelAnswer}[2]
  {\settowidth{\labelLength}{#1}%
   \addtolength{\labelLength}{0.25in}%
   \changetext{}{-\labelLength}{}{}{}%
   \noindent\fbox{\begin{minipage}[c]{\columnwidth}#2\end{minipage}}%
   \marginpar{\fbox{#1}}%

   % We put the blank space above in order to make sure this
   % \marginpar gets correctly placed.
   \changetext{}{+\labelLength}{}{}{}}%

\setcounter{secnumdepth}{0}
\newcommand{\homeworkProblemName}{}%
\newcounter{homeworkProblemCounter}%
\newenvironment{homeworkProblem}[1][Problem \arabic{homeworkProblemCounter}]%
  {\stepcounter{homeworkProblemCounter}%
   \renewcommand{\homeworkProblemName}{#1}%
   \section{\homeworkProblemName}%
   \enterProblemHeader{\homeworkProblemName}}%
  {\exitProblemHeader{\homeworkProblemName}}%

\newcommand{\problemAnswer}[1]
  {\noindent\fbox{\begin{minipage}[c]{\columnwidth}#1\end{minipage}}}%

\newcommand{\problemLAnswer}[1]
  {\labelAnswer{\homeworkProblemName}{#1}}

\newcommand{\homeworkSectionName}{}%
\newlength{\homeworkSectionLabelLength}{}%
\newenvironment{homeworkSection}[1]%
  {% We put this space here to make sure we're not connected to the above.
   % Otherwise the changetext can do funny things to the other margin

   \renewcommand{\homeworkSectionName}{#1}%
   \settowidth{\homeworkSectionLabelLength}{\homeworkSectionName}%
   \addtolength{\homeworkSectionLabelLength}{0.25in}%
   \changetext{}{-\homeworkSectionLabelLength}{}{}{}%
   \subsection{\homeworkSectionName}%
   \enterProblemHeader{\homeworkProblemName\ [\homeworkSectionName]}}%
  {\enterProblemHeader{\homeworkProblemName}%

   % We put the blank space above in order to make sure this margin
   % change doesn't happen too soon (otherwise \sectionAnswer's can
   % get ugly about their \marginpar placement.
   \changetext{}{+\homeworkSectionLabelLength}{}{}{}}%

\newcommand{\sectionAnswer}[1]
  {% We put this space here to make sure we're disconnected from the previous
   % passage

   \noindent\fbox{\begin{minipage}[c]{\columnwidth}#1\end{minipage}}%
   \enterProblemHeader{\homeworkProblemName}\exitProblemHeader{\homeworkProblemName}%
   \marginpar{\fbox{\homeworkSectionName}}%

   % We put the blank space above in order to make sure this
   % \marginpar gets correctly placed.
   }%

%%%%%%%%%%%%%%%%%%%%%%%%%%%%%%%%%%%%%%%%%%%%%%%%%%%%%%%%%%%%%


%%%%%%%%%%%%%%%%%%%%%%%%%%%%%%%%%%%%%%%%%%%%%%%%%%%%%%%%%%%%%
% Make title
\title{\vspace{2in}\textmd{\textbf{\hmwkClass:\ \hmwkTitle}}\\\normalsize\vspace{0.1in}\small{Due\ on\ \hmwkDueDate}\\\vspace{0.1in}\large{}\vspace{3in}}
\date{}
%\author{\textbf{\hmwkAuthorName}}
%%%%%%%%%%%%%%%%%%%%%%%%%%%%%%%%%%%%%%%%%%%%%%%%%%%%%%%%%%%%%

\begin{document}
\begin{spacing}{1.0}
%\maketitle
\newpage

\section{Setting up github for grading \& commentary}
You should have a confirmed educational account and private repository for
your homework and exercises at this point.  If you do not, please go back through
the previous git tutorial and complete the necessary steps.

Go to your github home page and open up your ASTR2600 repository (you could get
there directly with \url{https://github.com/username/ASTR2600}).  Click on the
``Admin'' tab near the top right.  Select the ``Teams'' tab on the left side.
Select the team ``ASTR2600 Graders'' and add that team.

If you don't have a ``Teams'' tab on the left side, but you DO have a
``Collaborators'' tab, click that.  Add \texttt{cwedge90} and
\texttt{keflavich} as collaborators.


\section{Tutorial: Forking}

A great deal of modern programming practice involves collaboration with your
colleagues.  The internet has made this much easier than ever before, and there
are specific tools on the web that make direct collaboration pretty easy.

Most software seems to still work on the ``old model'' of software development.
If you're using your Mac or Windows machine and something goes wrong (crashes,
simply doesn't work) or you find there's a feature you really want but it
doesn't exist, you're out of luck: you just reboot the machine and go on.

The model for ``open source'' code, on the other hand, allows you to submit
``bug reports'' saying things like, ``Doing this causes the code to crash'' or
``I'd really like this feature.''   With tools like github and bitbucket (a
competitor), there is a concept of ``forks'' and ``pull requests'' - with
these, you can say ``I'd really like this feature, and I wrote some code that I
think does it, so you should include it in your application.''

Our goal is to do some work as a class.

\section{Fork ASTR2600}
Open up github and go to the ASTR2600 page: \url{https://github.com/ASTR2600f12/ASTR2600}.\\
Make sure you're logged in (it should show your username towards the top-right).\\
Then click the ``Fork'' button (also near the top right).  Enjoy the strange picture,
comment, or animation at this point.

Now, clone your new forked repository.  There should be a box in the center with 
``SSH'' highlighted and text saying:\\ 
\url{git@github.com:yourusername/ASTR2600.git}\\
It should say ``Read+Write access'' next to it.

Copy that URL and paste it to make the following command:\\
\verb|git clone git@github.com:yourusername/ASTR2600.git ASTR2600_fork|

You should now have access to the \texttt{snake.pro} code you saw last time we 
worked with git.

\section{Edit some code - comments!}

Find \texttt{snake.pro} in the new \verb|ASTR2600_fork| directory and open it
in an editor (e.g., \verb|gedit|).

Now, go through that file and put comments wherever they are appropriate.
For example, you might think it appropriate to add a commend at the top saying,
``This code tries to animate a snake.''  You might also consider adding another
comment before line 1 saying ``What does spawn do?''

Remember, in IDL, comments are preceded with the semicolon \verb|;| character.

Continue through the code, adding comments where you don't understand things.
Where you DO understand what's going on, use the comments to describe the code.

Save your work!  

When you're done with that, do a \verb|git commit -a| on the command line (not 
in IDL).  You'll need to include a commit message.  Then, \verb|git push| your
results to the remote repository.

\section{Pull Request}
Go back to the github webpage and navigate to your copy of \verb|snake.pro|.  Have a
look at it; make sure it has all the comments you put in.  If it doesn't, check
to make sure you saved, committed, and pushed.

Now, you can create a ``pull request''.  There is a button towards the top of the
screen that says ``Pull Request'' - click it.

You should be shown a ``diff'' between your fork and the original repository.
You should also be given space to enter some text describing your pull request.
Put something useful in there - e.g., ``commented snake.pro with questions about
what it does''.

I will be able to see the changes you've made and comment on your pull requests.
I'll also be able to make line-by-line comments on YOUR comments.  So it's a
great way to discuss code.

\section{tvcircle}
In the original version of \texttt{snake.pro}, it apparently lacked eyes (despite
my best efforts).  Look at the \texttt{tvcircle} documentation (you can find it
in the \texttt{tvcircle.pro} source code as comments at the top, or look at this
URL: \url{http://astro.uni-tuebingen.de/software/idl/astrolib/tv/tvcircle.html}).
See if you can use it to make eyes for the snake.

If you succeed in making snake eyes (or maybe just one eye, depending on your
perspective), commit that change and push it.  It will then be included in your
pull request.

In the next episode, we'll try to make badgers and mushrooms.

\section{Bonus} 
Try using the \verb|cursor| command to draw something.  If you
like what you've drawn, save the \verb|x| and \verb|y| arrays or add them to
your code and push the code to github.  (Note: don't try to add an IDL .sav
file to github; binaries are not meant to be kept in repositories).

\end{spacing}
\end{document}

