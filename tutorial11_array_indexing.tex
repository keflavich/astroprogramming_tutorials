%\documentclass[11pt,letterpaper,notitlepage,onesided]{tex/nwh_hw}
%\documentclass[11pt,letterpaper,notitlepage]{article}
\documentclass{article}
% Change "article" to "report" to get rid of page number on title page
\usepackage{amsmath,amsfonts,amsthm,amssymb}
\usepackage{setspace}
\usepackage{textcomp}
\usepackage{listings}
\lstset{basicstyle=\ttfamily} % <<< This line added
\lstset{upquote=true}
\lstset{breakatwhitespace=true}
\renewcommand{\ttdefault}{cmtt}

\usepackage{Tabbing}
\usepackage{textcomp}
\usepackage{fancyhdr}
\usepackage{lastpage}
\usepackage{extramarks}
\usepackage{chngpage}
\usepackage{soul,color}
\usepackage{graphicx,float,wrapfig}
\usepackage{parskip}
\usepackage[utf8]{inputenc}
\usepackage[T1]{fontenc}

% In case you need to adjust margins:
\topmargin=-0.45in      %
\evensidemargin=0in     %
\oddsidemargin=0in      %
\textwidth=6.5in        %
\textheight=9.0in       %
\headsep=0.25in         %

% Homework Specific Information
\newcommand{\hmwkTitle}{Tutorial: Array Indexing / Spectroscopy}
\newcommand{\hmwkDueDate}{DATE, 4:00 PM}
\newcommand{\hmwkClass}{ASTR 2600}
\newcommand{\hmwkClassTime}{4:00 PM T/Th}
\newcommand{\hmwkClassInstructor}{Adam Ginsburg}
\newcommand{\hmwkAuthorName}{Dewey Anderson}

% Setup the header and footer
\pagestyle{fancy}                                                       %
%\lhead{\hmwkAuthorName}                                                 %
\chead{\hmwkClass\: \hmwkTitle}  %
\rhead{\firstxmark}                                                     %
\lfoot{\lastxmark}                                                      %
\cfoot{}                                                                %
\rfoot{Page\ \thepage\ of\ \pageref{LastPage}}                          %
\renewcommand\headrulewidth{0.4pt}                                      %
\renewcommand\footrulewidth{0.4pt}                                      %

\usepackage[utf8]{inputenc}
\usepackage[unicode=true]{hyperref}
\hypersetup{breaklinks=true,
            bookmarks=true,
            pdfauthor={},
            pdftitle={Connecting to the cosmos computer from home using Microsoft Windows},
            colorlinks=true,
            urlcolor=blue,
            linkcolor=magenta,
            pdfborder={0 0 0}}

% This is used to trace down (pin point) problems
% in latexing a document:
%\tracingall

%%%%%%%%%%%%%%%%%%%%%%%%%%%%%%%%%%%%%%%%%%%%%%%%%%%%%%%%%%%%%
% Some tools
\newcommand{\enterProblemHeader}[1]{\nobreak\extramarks{#1}{#1 continued on next page\ldots}\nobreak%
                                    \nobreak\extramarks{#1 (continued)}{#1 continued on next page\ldots}\nobreak}%
\newcommand{\exitProblemHeader}[1]{\nobreak\extramarks{#1 (continued)}{#1 continued on next page\ldots}\nobreak%
                                   \nobreak\extramarks{#1}{}\nobreak}%

\newlength{\labelLength}
\newcommand{\labelAnswer}[2]
  {\settowidth{\labelLength}{#1}%
   \addtolength{\labelLength}{0.25in}%
   \changetext{}{-\labelLength}{}{}{}%
   \noindent\fbox{\begin{minipage}[c]{\columnwidth}#2\end{minipage}}%
   \marginpar{\fbox{#1}}%

   % We put the blank space above in order to make sure this
   % \marginpar gets correctly placed.
   \changetext{}{+\labelLength}{}{}{}}%

\setcounter{secnumdepth}{0}
\newcommand{\homeworkProblemName}{}%
\newcounter{homeworkProblemCounter}%
\newenvironment{homeworkProblem}[1][Problem \arabic{homeworkProblemCounter}]%
  {\stepcounter{homeworkProblemCounter}%
   \renewcommand{\homeworkProblemName}{#1}%
   \section{\homeworkProblemName}%
   \enterProblemHeader{\homeworkProblemName}}%
  {\exitProblemHeader{\homeworkProblemName}}%

\newcommand{\problemAnswer}[1]
  {\noindent\fbox{\begin{minipage}[c]{\columnwidth}#1\end{minipage}}}%

\newcommand{\problemLAnswer}[1]
  {\labelAnswer{\homeworkProblemName}{#1}}

\newcommand{\homeworkSectionName}{}%
\newlength{\homeworkSectionLabelLength}{}%
\newenvironment{homeworkSection}[1]%
  {% We put this space here to make sure we're not connected to the above.
   % Otherwise the changetext can do funny things to the other margin

   \renewcommand{\homeworkSectionName}{#1}%
   \settowidth{\homeworkSectionLabelLength}{\homeworkSectionName}%
   \addtolength{\homeworkSectionLabelLength}{0.25in}%
   \changetext{}{-\homeworkSectionLabelLength}{}{}{}%
   \subsection{\homeworkSectionName}%
   \enterProblemHeader{\homeworkProblemName\ [\homeworkSectionName]}}%
  {\enterProblemHeader{\homeworkProblemName}%

   % We put the blank space above in order to make sure this margin
   % change doesn't happen too soon (otherwise \sectionAnswer's can
   % get ugly about their \marginpar placement.
   \changetext{}{+\homeworkSectionLabelLength}{}{}{}}%

\newcommand{\sectionAnswer}[1]
  {% We put this space here to make sure we're disconnected from the previous
   % passage

   \noindent\fbox{\begin{minipage}[c]{\columnwidth}#1\end{minipage}}%
   \enterProblemHeader{\homeworkProblemName}\exitProblemHeader{\homeworkProblemName}%
   \marginpar{\fbox{\homeworkSectionName}}%

   % We put the blank space above in order to make sure this
   % \marginpar gets correctly placed.
   }%

%%%%%%%%%%%%%%%%%%%%%%%%%%%%%%%%%%%%%%%%%%%%%%%%%%%%%%%%%%%%%


%%%%%%%%%%%%%%%%%%%%%%%%%%%%%%%%%%%%%%%%%%%%%%%%%%%%%%%%%%%%%
% Make title
\title{\vspace{2in}\textmd{\textbf{\hmwkClass:\ \hmwkTitle}}\\\normalsize\vspace{0.1in}\small{Due\ on\ \hmwkDueDate}\\\vspace{0.1in}\large{}\vspace{3in}}
\date{}
%\author{\textbf{\hmwkAuthorName}}
%%%%%%%%%%%%%%%%%%%%%%%%%%%%%%%%%%%%%%%%%%%%%%%%%%%%%%%%%%%%%

\begin{document}
\begin{spacing}{1.0}
%\maketitle
\newpage



\section{Tutorial: Array Indexing and Spectroscopy}

We will examine the properties of a spectrum!

\section{Read the data}
There is a spectrum file sitting in the shared data directory:\\
\verb|/home/shared/astr2600/data/spectrum.txt|\\
The wavelength is in units of Angstroms (\AA).  The flux units are
not specified.


Open this in a text editor (\verb|gedit|, \verb|vim|, or whatever).\\
Examine it.  How many lines are in the file?  \underline{\hspace{3cm}}
How many columns? \underline{\hspace{3cm}}

Keep that path handy as you open IDL.

In IDL, find out how many lines there are in the file using \verb|file_lines|:\\
\begin{lstlisting}
nlines = file_lines('/home/shared/astr2600/data/spectrum.txt')
\end{lstlisting}

Check to make sure that agrees with the number you found above.

Now, make two variables to store the wavelength and flux arrays.
Which column is the flux and which is the wavelength? \\
Flux Column \#: \underline{\hspace{3cm}}
Wavelength Column \#: \underline{\hspace{3cm}}
\begin{verbatim}
wavelength = fltarr(nlines)
flux = fltarr(nlines)
; replace ncols with the number of columns
xy = fltarr(ncols, nlines)
\end{verbatim}

Now, read the data in from the file.  
\begin{lstlisting}
openr,lun,'/home/shared/astr2600/data/spectrum.txt',/get_lun
readf,lun,xy
wavelength = xy[0,*]
flux = xy[1,*]
free_lun,lun
\end{lstlisting}

Remember what all that does?  Just to see some mistakes you could have made, let's try plotting some things.

\verb|plot,xy|

What does this look like?  Why?  (just think about it.  Maybe print out a few elements of \verb|xy| to check)

\section{Examine the data}
Zoom in on the line centered at 1330 angstroms.

First, set the width of the zoom and the center:
\begin{lstlisting}
zoomwidth=10
center=1330
\end{lstlisting}

Then, set your upper and lower bounds: \nobreak
\begin{verbatim}
lowerbound = center - zoomwidth/2
upperbound = center + zoomwidth/2
\end{verbatim}

Now, use the \verb|where| function to pick out the region you want to plot:
\begin{verbatim}
zoomregion = where( (wavelength gt upperbound) and (wavelength lt lowerbound), npts )
\end{verbatim}
% zoomregion = where( (wavelength lt upperbound) and (wavelength gt lowerbound), npts )

What is \verb|npts|?  \underline{\hspace{3cm}}\\
Why?  What did we do wrong?  Fix it!

Now plot up your results:
\begin{verbatim}
plot,wavelength[zoomregion],flux
\end{verbatim}
% plot,wavelength[zoomregion],flux[zoomregion]

Does this look like the region centered around 1330 that we had looked at before?  \\
Why not? What did we do wrong this time?  Again, fix it!

\section{Measurements}
We shall measure the data.

First, what is $\Delta \lambda$?  I expect you to be able to get this, but ask a
neighbor if you're not sure how (don't just get the answer, understand the
method)\\
$\Delta \lambda=$\underline{\hspace{3cm}}

What is the continuum?  (approximately, get it from the plot) \\
$continuum\approx$\underline{\hspace{3cm}}

Now, determine how much light is absorbed.  There are multiple ways you can do
this.  One way is to determine the ``absorption spectrum'' as we did in the
lecture example.  Another would be to determine the ``area under the continuum''
and the ``area under the spectrum'' and take the difference.  
Pick one of these methods - or another if you can think of any - and determine the
absorbed flux.

Absorbed flux: \underline{\hspace{3cm}}

\section{Equivalent Width}
What is the equivalent width of the line?  Recall that equivalent width means
``the width of the box that has the same area as the absorption line.''  It is
a box going from the bottom to the top of the plot.
$EQW=$\underline{\hspace{3cm}}\AA

Overplot a box centered on the line center with the correct height and width.
Use \verb|color='00AAFF'x| (orange).

To plot a box, you need to draw 4 lines.  You can determine what the 4 corners are:\\
$x1=$\underline{\hspace{3cm}}  $y1=$\underline{\hspace{3cm}} \\
$x2=$\underline{\hspace{3cm}}  $y2=$\underline{\hspace{3cm}} \\
$x3=$\underline{\hspace{3cm}}  $y3=$\underline{\hspace{3cm}} \\
$x4=$\underline{\hspace{3cm}}  $y4=$\underline{\hspace{3cm}} \\

Then, you need an array with \emph{5} elements!  (each pair of corners gets you one line, but plot
doesn't automatically draw from the last back to the first)\\
\verb|oplot,[x1,x2,x3,x4,x1],[y1,y2,y3,y4,y1],color='00AAFF'x|

You've now measured and visualized Equivalent Width.  We'll do more with it eventually.

\end{spacing}
\end{document}

