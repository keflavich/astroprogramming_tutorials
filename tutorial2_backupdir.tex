%\documentclass[11pt,letterpaper,notitlepage,onesided]{tex/nwh_hw}
%\documentclass[11pt,letterpaper,notitlepage]{article}
\documentclass{article}
% Change "article" to "report" to get rid of page number on title page
\usepackage{amsmath,amsfonts,amsthm,amssymb}
\usepackage{setspace}
\usepackage{listings}
\usepackage{Tabbing}
\usepackage{textcomp}
\usepackage{fancyhdr}
\usepackage{lastpage}
\usepackage{extramarks}
\usepackage{chngpage}
\usepackage{soul,color}
\usepackage{graphicx,float,wrapfig}
\usepackage{parskip}
\usepackage[utf8]{inputenc}

% In case you need to adjust margins:
\topmargin=-0.45in      %
\evensidemargin=0in     %
\oddsidemargin=0in      %
\textwidth=6.5in        %
\textheight=9.0in       %
\headsep=0.25in         %

% Homework Specific Information
\newcommand{\hmwkTitle}{TUTORIAL: Making Backups}
\newcommand{\hmwkDueDate}{DATE, 4:00 PM}
\newcommand{\hmwkClass}{ASTR 2600}
\newcommand{\hmwkClassTime}{4:00 PM T/Th}
\newcommand{\hmwkClassInstructor}{Adam Ginsburg}

% Setup the header and footer
\pagestyle{fancy}                                                       %
%\lhead{\hmwkAuthorName}                                                 %
\chead{\hmwkClass\: \hmwkTitle}  %
\rhead{\firstxmark}                                                     %
\lfoot{\lastxmark}                                                      %
\cfoot{}                                                                %
\rfoot{Page\ \thepage\ of\ \pageref{LastPage}}                          %
\renewcommand\headrulewidth{0.4pt}                                      %
\renewcommand\footrulewidth{0.4pt}                                      %

\usepackage[utf8]{inputenc}
\usepackage[unicode=true]{hyperref}
\hypersetup{breaklinks=true,
            bookmarks=true,
            pdfauthor={},
            pdftitle={Connecting to the cosmos computer from home using Microsoft Windows},
            colorlinks=true,
            urlcolor=blue,
            linkcolor=magenta,
            pdfborder={0 0 0}}

% This is used to trace down (pin point) problems
% in latexing a document:
%\tracingall

%%%%%%%%%%%%%%%%%%%%%%%%%%%%%%%%%%%%%%%%%%%%%%%%%%%%%%%%%%%%%
% Some tools
\newcommand{\enterProblemHeader}[1]{\nobreak\extramarks{#1}{#1 continued on next page\ldots}\nobreak%
                                    \nobreak\extramarks{#1 (continued)}{#1 continued on next page\ldots}\nobreak}%
\newcommand{\exitProblemHeader}[1]{\nobreak\extramarks{#1 (continued)}{#1 continued on next page\ldots}\nobreak%
                                   \nobreak\extramarks{#1}{}\nobreak}%

\newlength{\labelLength}
\newcommand{\labelAnswer}[2]
  {\settowidth{\labelLength}{#1}%
   \addtolength{\labelLength}{0.25in}%
   \changetext{}{-\labelLength}{}{}{}%
   \noindent\fbox{\begin{minipage}[c]{\columnwidth}#2\end{minipage}}%
   \marginpar{\fbox{#1}}%

   % We put the blank space above in order to make sure this
   % \marginpar gets correctly placed.
   \changetext{}{+\labelLength}{}{}{}}%

\setcounter{secnumdepth}{0}
\newcommand{\homeworkProblemName}{}%
\newcounter{homeworkProblemCounter}%
\newenvironment{homeworkProblem}[1][Problem \arabic{homeworkProblemCounter}]%
  {\stepcounter{homeworkProblemCounter}%
   \renewcommand{\homeworkProblemName}{#1}%
   \section{\homeworkProblemName}%
   \enterProblemHeader{\homeworkProblemName}}%
  {\exitProblemHeader{\homeworkProblemName}}%

\newcommand{\problemAnswer}[1]
  {\noindent\fbox{\begin{minipage}[c]{\columnwidth}#1\end{minipage}}}%

\newcommand{\problemLAnswer}[1]
  {\labelAnswer{\homeworkProblemName}{#1}}

\newcommand{\homeworkSectionName}{}%
\newlength{\homeworkSectionLabelLength}{}%
\newenvironment{homeworkSection}[1]%
  {% We put this space here to make sure we're not connected to the above.
   % Otherwise the changetext can do funny things to the other margin

   \renewcommand{\homeworkSectionName}{#1}%
   \settowidth{\homeworkSectionLabelLength}{\homeworkSectionName}%
   \addtolength{\homeworkSectionLabelLength}{0.25in}%
   \changetext{}{-\homeworkSectionLabelLength}{}{}{}%
   \subsection{\homeworkSectionName}%
   \enterProblemHeader{\homeworkProblemName\ [\homeworkSectionName]}}%
  {\enterProblemHeader{\homeworkProblemName}%

   % We put the blank space above in order to make sure this margin
   % change doesn't happen too soon (otherwise \sectionAnswer's can
   % get ugly about their \marginpar placement.
   \changetext{}{+\homeworkSectionLabelLength}{}{}{}}%

\newcommand{\sectionAnswer}[1]
  {% We put this space here to make sure we're disconnected from the previous
   % passage

   \noindent\fbox{\begin{minipage}[c]{\columnwidth}#1\end{minipage}}%
   \enterProblemHeader{\homeworkProblemName}\exitProblemHeader{\homeworkProblemName}%
   \marginpar{\fbox{\homeworkSectionName}}%

   % We put the blank space above in order to make sure this
   % \marginpar gets correctly placed.
   }%

%%%%%%%%%%%%%%%%%%%%%%%%%%%%%%%%%%%%%%%%%%%%%%%%%%%%%%%%%%%%%


%%%%%%%%%%%%%%%%%%%%%%%%%%%%%%%%%%%%%%%%%%%%%%%%%%%%%%%%%%%%%
% Make title
\title{\vspace{2in}\textmd{\textbf{\hmwkClass:\ \hmwkTitle}}\\\normalsize\vspace{0.1in}\small{Due\ on\ \hmwkDueDate}\\\vspace{0.1in}\large{}\vspace{3in}}
\date{}
%\author{\textbf{\hmwkAuthorName}}
%%%%%%%%%%%%%%%%%%%%%%%%%%%%%%%%%%%%%%%%%%%%%%%%%%%%%%%%%%%%%

\begin{document}
\begin{spacing}{1.0}
%\maketitle
\newpage



\section{Make a Backup Directory}

This tutorial is to set up a ``backup directory'' for you to copy files into in
order to prevent accidentally losing data.

We'll use the unix commands \verb|cp|, \verb|mkdir|, \verb|cd|, \verb|ls|,
\verb|pwd|, \verb|echo|, \verb|cat|.

\subsection{Make Backups}

\begin{enumerate}
    \item Enter \verb|pwd|.  \verb|pwd| stands for ``present working
        directory''; it is like asking ``Where am I now?''
        You should be in your home directory.
    \item Even if you are already in your home directory, enter \verb|cd|.
        Just \verb|cd|, with nothing after it, moves you to your home
        directory.  So does \verb|cd ~|.
    \item Enter \verb|ls| to ``list'' the contents of your home directory
    \item Make a new directory called ``backup'': \verb|mkdir backup|
    \item Find all \verb|.pro| files you've made so far: \verb|ls *pro|.
        ``*'' is called a ``wildcard''.  It represents any number or letter, so
        \verb|*pro| will find all files that end with the letters \verb|pro|
    \item Copy all of your \verb|.pro| files into your \verb|backup/| directory:
        \verb|cp *pro backup/|
\end{enumerate}

We just demonstrated some of the power of the UNIX operating system.  Now we'll
explore some of the dangers.

\subsection{Screw up a backup}

\begin{enumerate}
    \item Enter \verb|echo "; This text will be lost" > temporary.pro|.  This
        is a complicated command involving a ``redirect''.  \verb|echo| is like
        the print command in IDL; it is harmless.  \verb|>| means that instead
        of being ``printed to the screen'', the output of \verb|echo| is
        redirected to a file.
    \item Now copy \verb|temporary.pro| to \verb|backup/|
    \item Verify that the copy succeeded with \verb|ls backup/|
    \item Verify that the text is still what you wrote: open \verb|temporary.pro|
        in \verb|gedit| with \verb|gedit backup/temporary.pro &|.
    \item Once you've verified that the text is there, close \verb|gedit|.
    \item One more verification: \verb|cat backup/temporary.pro|
    \item Now let's overwrite \verb|temporary.pro| with something useless:
        \verb|echo "This text is junk" > temporary.pro|.
    \item You can verify that \verb|backup/temporary.pro| is still safe with
        \verb|cat backup/temporary.pro|
    \item Now, repeat the command from above, \verb|cp *pro backup/|
    \item Look at the contents of \verb|backup/temporary.pro|: \verb|cat backup/temporary.pro|
    \item We never wanted either \verb|temporary.pro| in the first place.
        Remove both with the commands \verb|rm temporary.pro| and \verb|rm
        backup/temporary.pro|.  \emph{BE CAREFUL not to use wildcards for these
        operations!}
\end{enumerate}

Short story: It's really easy to overwrite files.  It's also easy to do
unexpected things when using shortcuts like wildcards.  Be careful!

This little exercise is a lead-in to more advanced code backup techniques,
specifically something called “version control”.  If you're interested in
learning more about it now, google \texttt{git} or \texttt{mercurial} and have
a look at \url{http://hginit.com}.

\end{spacing}
\end{document}

