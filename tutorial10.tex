%\documentclass[11pt,letterpaper,notitlepage,onesided]{tex/nwh_hw}
%\documentclass[11pt,letterpaper,notitlepage]{article}
\documentclass{article}
% Change "article" to "report" to get rid of page number on title page
\usepackage{amsmath,amsfonts,amsthm,amssymb}
\usepackage{setspace}
\usepackage{textcomp}
\usepackage{listings}
\lstset{basicstyle=\ttfamily} % <<< This line added
\lstset{upquote=true}
\lstset{breakatwhitespace=true}
\renewcommand{\ttdefault}{cmtt}

\usepackage{Tabbing}
\usepackage{textcomp}
\usepackage{fancyhdr}
\usepackage{lastpage}
\usepackage{extramarks}
\usepackage{chngpage}
\usepackage{soul,color}
\usepackage{graphicx,float,wrapfig}
\usepackage{parskip}
\usepackage[utf8]{inputenc}
\usepackage[T1]{fontenc}

% In case you need to adjust margins:
\topmargin=-0.45in      %
\evensidemargin=0in     %
\oddsidemargin=0in      %
\textwidth=6.5in        %
\textheight=9.0in       %
\headsep=0.25in         %

% Homework Specific Information
\newcommand{\hmwkTitle}{Tutorial: Getting started with Version Control}
\newcommand{\hmwkDueDate}{DATE, 4:00 PM}
\newcommand{\hmwkClass}{ASTR 2600}
\newcommand{\hmwkClassTime}{4:00 PM T/Th}
\newcommand{\hmwkClassInstructor}{Adam Ginsburg}
\newcommand{\hmwkAuthorName}{Dewey Anderson}

% Setup the header and footer
\pagestyle{fancy}                                                       %
%\lhead{\hmwkAuthorName}                                                 %
\chead{\hmwkClass\: \hmwkTitle}  %
\rhead{\firstxmark}                                                     %
\lfoot{\lastxmark}                                                      %
\cfoot{}                                                                %
\rfoot{Page\ \thepage\ of\ \pageref{LastPage}}                          %
\renewcommand\headrulewidth{0.4pt}                                      %
\renewcommand\footrulewidth{0.4pt}                                      %

\usepackage[utf8]{inputenc}
\usepackage[unicode=true]{hyperref}
\hypersetup{breaklinks=true,
            bookmarks=true,
            pdfauthor={},
            pdftitle={Connecting to the cosmos computer from home using Microsoft Windows},
            colorlinks=true,
            urlcolor=blue,
            linkcolor=magenta,
            pdfborder={0 0 0}}

% This is used to trace down (pin point) problems
% in latexing a document:
%\tracingall

%%%%%%%%%%%%%%%%%%%%%%%%%%%%%%%%%%%%%%%%%%%%%%%%%%%%%%%%%%%%%
% Some tools
\newcommand{\enterProblemHeader}[1]{\nobreak\extramarks{#1}{#1 continued on next page\ldots}\nobreak%
                                    \nobreak\extramarks{#1 (continued)}{#1 continued on next page\ldots}\nobreak}%
\newcommand{\exitProblemHeader}[1]{\nobreak\extramarks{#1 (continued)}{#1 continued on next page\ldots}\nobreak%
                                   \nobreak\extramarks{#1}{}\nobreak}%

\newlength{\labelLength}
\newcommand{\labelAnswer}[2]
  {\settowidth{\labelLength}{#1}%
   \addtolength{\labelLength}{0.25in}%
   \changetext{}{-\labelLength}{}{}{}%
   \noindent\fbox{\begin{minipage}[c]{\columnwidth}#2\end{minipage}}%
   \marginpar{\fbox{#1}}%

   % We put the blank space above in order to make sure this
   % \marginpar gets correctly placed.
   \changetext{}{+\labelLength}{}{}{}}%

\setcounter{secnumdepth}{0}
\newcommand{\homeworkProblemName}{}%
\newcounter{homeworkProblemCounter}%
\newenvironment{homeworkProblem}[1][Problem \arabic{homeworkProblemCounter}]%
  {\stepcounter{homeworkProblemCounter}%
   \renewcommand{\homeworkProblemName}{#1}%
   \section{\homeworkProblemName}%
   \enterProblemHeader{\homeworkProblemName}}%
  {\exitProblemHeader{\homeworkProblemName}}%

\newcommand{\problemAnswer}[1]
  {\noindent\fbox{\begin{minipage}[c]{\columnwidth}#1\end{minipage}}}%

\newcommand{\problemLAnswer}[1]
  {\labelAnswer{\homeworkProblemName}{#1}}

\newcommand{\homeworkSectionName}{}%
\newlength{\homeworkSectionLabelLength}{}%
\newenvironment{homeworkSection}[1]%
  {% We put this space here to make sure we're not connected to the above.
   % Otherwise the changetext can do funny things to the other margin

   \renewcommand{\homeworkSectionName}{#1}%
   \settowidth{\homeworkSectionLabelLength}{\homeworkSectionName}%
   \addtolength{\homeworkSectionLabelLength}{0.25in}%
   \changetext{}{-\homeworkSectionLabelLength}{}{}{}%
   \subsection{\homeworkSectionName}%
   \enterProblemHeader{\homeworkProblemName\ [\homeworkSectionName]}}%
  {\enterProblemHeader{\homeworkProblemName}%

   % We put the blank space above in order to make sure this margin
   % change doesn't happen too soon (otherwise \sectionAnswer's can
   % get ugly about their \marginpar placement.
   \changetext{}{+\homeworkSectionLabelLength}{}{}{}}%

\newcommand{\sectionAnswer}[1]
  {% We put this space here to make sure we're disconnected from the previous
   % passage

   \noindent\fbox{\begin{minipage}[c]{\columnwidth}#1\end{minipage}}%
   \enterProblemHeader{\homeworkProblemName}\exitProblemHeader{\homeworkProblemName}%
   \marginpar{\fbox{\homeworkSectionName}}%

   % We put the blank space above in order to make sure this
   % \marginpar gets correctly placed.
   }%

%%%%%%%%%%%%%%%%%%%%%%%%%%%%%%%%%%%%%%%%%%%%%%%%%%%%%%%%%%%%%


%%%%%%%%%%%%%%%%%%%%%%%%%%%%%%%%%%%%%%%%%%%%%%%%%%%%%%%%%%%%%
% Make title
\title{\vspace{2in}\textmd{\textbf{\hmwkClass:\ \hmwkTitle}}\\\normalsize\vspace{0.1in}\small{Due\ on\ \hmwkDueDate}\\\vspace{0.1in}\large{}\vspace{3in}}
\date{}
%\author{\textbf{\hmwkAuthorName}}
%%%%%%%%%%%%%%%%%%%%%%%%%%%%%%%%%%%%%%%%%%%%%%%%%%%%%%%%%%%%%

\begin{document}
\begin{spacing}{1.0}
%\maketitle
\newpage



\section{Tutorial: Version Control}

We're taking a huge step into a much more advanced realm of program
development.

Almost all programmers outside of the sciences do their programming
collaboratively.  Recently, the same has become true of programmers within
science as well.

Collaborative programming is when you allow others to help you with your code,
or as a group you combine your efforts to write one big code instead of many
separate ones.

The simplest way to do this is for each individual to write their own code,
pass it around, and have colleagues write down their own comments and
suggestions on how to change or improve the code.  While simple, this is
horribly inefficient: what if the only change your code needed was the addition
of a semicolon where you accidentally left one out?  There are many other
reasons this method isn't great (but it is still the preferred method of editing
papers).

There is a powerful unix tool called \texttt{diff} that allows you to directly
compare files.  The \texttt{diff} command goes through a pair of text files
line-by-line identifying differences on each line, but ignoring lines that are
the same.

There is a whole huge framework of tools built on top of \texttt{diff}.  We're
going to skip over a lot of the why and how for those so we can get to more
immediately useful stuff.

\section{git}
\texttt{git} is a ``distributed version control system''.  Those words contain a
lot of layered meaning, but for now you need to know that \texttt{git}:
\begin{enumerate}
    \item is a way to keep track of edits to your code
    \item provides local (and remote) backups of your code
    \item is a great way to share code
\end{enumerate}
There are other similar codes: \texttt{hg} (mercurial) is basically the same
thing, while \texttt{svn} (subversion) is a little different but similar.
We're using \texttt{git} simply because \texttt{github} gave us an educational
account.

You're going to make a `git' account for your own use and for this class.

Go to \url{http://github.com} and click the `Signup and Pricing' button.
Then, click the `Create a free account' button.

It will ask you for a username, e-mail address, and password.  For your
username, pick anything you'd like (but be aware, it will be visible to the
internet, including me).  For your e-mail address, use your
\texttt{@colorado.edu} address - this is important, as it will allow you access
to some free \texttt{.edu} features that are not otherwise available.

E-mail your account name to Adam (\url{astr2600@gmail.com}).  

%Once you've logged in, go to \url{https://github.com/edu}
Once you've logged in, click on the “Account Settings” button (top right, looks
like a screwdriver + wrench).  Click on the “Emails” tab on the left side.
Then, click the “Verify” button.  You will receive an e-mail sometime soon
confirming that you are indeed a student and giving you free access to private
repositories (we'll talk about what those are).

Go back to the github homepage \url{http://github.com}.  You should be logged in.
On the top-left, there should be a button with your username and a drop-down
triangle to the right side.  Click on that, and select “ASTR2600f12”.

On the right side, there is a region labeled `Repositories'.  Under `All
Repositories', click the \texttt{ASTR2600f12/ASTR2600} link.  This is the
“repository page” where all the code is hosted.  But that's not really the 
point, yet...

\section{Making an SSH key}

SSH stands for ``secure shell''.  It is a cryptographic network protocol, which
means it's a way to communicate between computers securely so that no one can
see what you send or receive.

When you're sending data to and receiving data from github, git uses the ``ssh
protocol'' to protect your data.

For this process to work properly and easily, you need a way for the remote repository
(the github website) to know that you, and not someone else pretending to be you, are
asking for the data.

So, we're going to make a ``public key'' to send to github.  Encryption is a very complicated
topic on its own, so we're not going to discuss it in detail - this will just be a simple
``how-to''.

Run the command \verb|ssh-keygen|.  When it prompts you to ``Enter file in
which to save the key'', press \texttt{enter} to accept the default.  Press
\texttt{enter} twice more to accept ``no passphrase''.

To see the contents of the public key, use the cat command:\\
\verb|cat ~/.ssh/id_rsa.pub|

Now load up the github web page again.  Go to ``settings'', then select the
``SSH Keys'' tab on the left side.  Click ``Add SSH Key'' in the top right.

Copy and paste the text shown into the ``Key'' text box.  You can use any title
you'd like, but I recommend calling it ``cosmos'' since that's the name of the
computer it represents.

Now you should be able to send/receive data from cosmos very easily.

\section{git and cloning}

Back at your terminal, type the following command (note that you can copy \&
paste the URL from the web page that is currently open):

\verb|git clone git@github.com:ASTR2600f12/ASTR2600.git ASTR2600-examples-git|

\begin{itemize}
    \item \texttt{git} is the command for the version control program.  All commands
        using \texttt{git} will start with this.
    \item \texttt{clone} tells git that you want to `clone', i.e. create an
        exact copy of, the repository
    \item \texttt{git@github.com:ASTR2600f12/ASTR2600.git} is the full URL of
        the repository, with special weird syntax.  Don't worry about it; I
        always copy \& paste this stuff myself.
    \item \texttt{ASTR2600-examples-git} is the name of the directory we have copied everything into
\end{itemize}

You may see a message like:
\verb|The authenticity of host 'github.com (207.97.227.239)' can't be established.|
Answer “yes”.  Technically, you should make sure that it's right before saying
yes, but unless someone is explicitly trying to attack our computer (cosmos) right now,
it should be fine.

Now, \texttt{cd} into the \verb|ASTR2600-examples-git/sample_code| directory.  Do an
\verb|ls| to show the contents.  You should see a directory, \texttt{coyote/},
a file \texttt{snake.pro}, and a file \texttt{tvcircle.pro}.

For kicks, open up IDL and do the following:

\verb|.r snake.pro|

If all goes well, you should see an (extremely poor) animation of something vaguely
resembling a snake.  Next up are badgers and mushrooms (ignore this sentence).

So far, you have created an account for yourself and acquired some code that I wrote.
This is a great way to share code, but it is only the tip of the iceberg.

Quit out of IDL before going to the next section.

\section{Create your own local repository}

This is to create a version control system for your own work.  It will include “incremental
history” and backups.  That means that, if you use the version control system correctly,
you will have a history of your work and all the changes you've made, AND you'll have a nice
set of backups.

Change directories to your ASTR2600 directory (should be \verb|cd ~/ASTR2600/|).
Once there, type the command:\\
\verb|git init|

You should see a message like:\\
\verb|Initialized empty Git repository in /home/astr/grad/ginsbura/ASTR2600/.git/|

Next, we'll add all of your files and folders to the repository:\\
\verb|git add *|

To see what the \texttt{add} command has done, use the status command:\\
\verb|git status|

You should see a list of all of your assignment files, preceded by the text \verb|new file:|.

Now we need to pick an editor.  If you like the editor \verb|gedit|, use it for
the following command, otherwise use \verb|vi|, \verb|nano|, or whatever you
prefer in its place.\\
\verb|git config core.editor gedit|

We'll now “commit changes” to the repository.  Up until now, you haven't
created a backup of your files, but once you complete this command, they will
all be stored \& backed up in the local repository.

\verb|git commit|

This will pop up a gedit window with all of the text you saw in 
\verb|git status|, plus room at the top for you to enter a message.  Do that!  Your
message should include a brief note about what you're adding (e.g.,
assignments, assignment numbers) and a statement that this is the first commit.  
In general, you want these notes to be long and verbose!

When you're done entering your message, save and quit out of \verb|gedit| or
whichever editor you're using.  You'll get a long verbose message, but short
story is your code has been added.

\section{Create your own repository on github}

In your github browser window, click the icon with a plus on it (I think it's a
line printer or something?  not sure); if you hover your mouse over the icon it
should say “Create a New Repo”).

Make the repository name “\verb|ASTR2600_assignments|”.  You can add something
in the description if you'd like.  Make sure you select the \emph{Private}
button.  If it asks you to pay, stop here and skip to the next section...

% Click the “Initialize this repository with a README” button.

Click “Create repository” [if it asked you to pay, this won't work]


\section{TO DO once you've gotten confirmed with an educational account}
Before doing this section, you must create a private repository (see ``Create
your own repository on github'' above).

Once you've successfully created your remote repository, you need to tell your git repository
on cosmos where it is. Your command should look like this (replace username with your username):

\verb|git remote add origin git@github.com:username/ASTR2600_assignments.git|

This tells git to add a new remote repository called \texttt{origin} with the URL
specified.

You can now “push your changes upstream”, a strange turn of phrase meaning that
you can take your local code (the stuff on cosmos) and “push” (upload) it to
the remote host on github, where it will be stored and you can access it from
anywhere.

\verb|git push -u origin master|

This command is a touch confusing, and it does a lot of things.  The
\texttt{-u} option requires more explaining than I want to do here.
\texttt{origin} indicates the name of the remote repository: the location
you're pushing (uploading) to.  \texttt{master} is the name of the local
“branch” you're operating on.  Branches are very useful things that grow from
trees, but that analogy deserves some lecture time.

% http://stackoverflow.com/questions/4658606/import-existing-source-code-to-github

\section{What next?}
From now on, any time you make changes to any of your files, you should
use \verb|git commit -a| to back up your files\footnote{The \texttt{-a} is required
to tell git to include your changes; it doesn't do that by default}.  Then use \verb|git push|
to send your data to the remote repository.

You can - and should - browse around github to look at the source code as it is
stored on the server.  See if you can figure out how to look at \texttt{diff}s
(we'll do it together at some point).

Any time you create a new file or folder, you will need to do a \verb|git add|
to add the new files, then do a \verb|git commit| to store them.


\end{spacing}
\end{document}

