%\documentclass[11pt,letterpaper,notitlepage,onesided]{tex/nwh_hw}
%\documentclass[11pt,letterpaper,notitlepage]{article}
\documentclass{article}
% Change "article" to "report" to get rid of page number on title page
\usepackage{amsmath,amsfonts,amsthm,amssymb}
\usepackage{setspace}
\usepackage{textcomp}
\usepackage{listings}
\lstset{basicstyle=\ttfamily} % <<< This line added
\lstset{upquote=true}
\lstset{breakatwhitespace=true}
\renewcommand{\ttdefault}{cmtt}

\usepackage{Tabbing}
\usepackage{textcomp}
\usepackage{fancyhdr}
\usepackage{lastpage}
\usepackage{extramarks}
\usepackage{chngpage}
\usepackage{soul,color}
\usepackage{graphicx,float,wrapfig}
\usepackage{parskip}
\usepackage[utf8]{inputenc}
\usepackage[T1]{fontenc}

% In case you need to adjust margins:
\topmargin=-0.45in      %
\evensidemargin=0in     %
\oddsidemargin=0in      %
\textwidth=6.5in        %
\textheight=9.0in       %
\headsep=0.25in         %

% Homework Specific Information
\newcommand{\hmwkTitle}{Tutorial: Git Setup}
\newcommand{\hmwkDueDate}{DATE, 4:00 PM}
\newcommand{\hmwkClass}{ASTR 2600}
\newcommand{\hmwkClassTime}{4:00 PM T/Th}
\newcommand{\hmwkClassInstructor}{Adam Ginsburg}
\newcommand{\hmwkAuthorName}{Dewey Anderson}

% Setup the header and footer
\pagestyle{fancy}                                                       %
%\lhead{\hmwkAuthorName}                                                 %
\chead{\hmwkClass\: \hmwkTitle}  %
\rhead{\firstxmark}                                                     %
\lfoot{\lastxmark}                                                      %
\cfoot{}                                                                %
\rfoot{Page\ \thepage\ of\ \pageref{LastPage}}                          %
\renewcommand\headrulewidth{0.4pt}                                      %
\renewcommand\footrulewidth{0.4pt}                                      %

\usepackage[utf8]{inputenc}
\usepackage[unicode=true]{hyperref}
\hypersetup{breaklinks=true,
            bookmarks=true,
            pdfauthor={},
            pdftitle={Connecting to the cosmos computer from home using Microsoft Windows},
            colorlinks=true,
            urlcolor=blue,
            linkcolor=magenta,
            pdfborder={0 0 0}}

% This is used to trace down (pin point) problems
% in latexing a document:
%\tracingall

%%%%%%%%%%%%%%%%%%%%%%%%%%%%%%%%%%%%%%%%%%%%%%%%%%%%%%%%%%%%%
% Some tools
\newcommand{\enterProblemHeader}[1]{\nobreak\extramarks{#1}{#1 continued on next page\ldots}\nobreak%
                                    \nobreak\extramarks{#1 (continued)}{#1 continued on next page\ldots}\nobreak}%
\newcommand{\exitProblemHeader}[1]{\nobreak\extramarks{#1 (continued)}{#1 continued on next page\ldots}\nobreak%
                                   \nobreak\extramarks{#1}{}\nobreak}%

\newlength{\labelLength}
\newcommand{\labelAnswer}[2]
  {\settowidth{\labelLength}{#1}%
   \addtolength{\labelLength}{0.25in}%
   \changetext{}{-\labelLength}{}{}{}%
   \noindent\fbox{\begin{minipage}[c]{\columnwidth}#2\end{minipage}}%
   \marginpar{\fbox{#1}}%

   % We put the blank space above in order to make sure this
   % \marginpar gets correctly placed.
   \changetext{}{+\labelLength}{}{}{}}%

%\setcounter{secnumdepth}{2}
\newcommand{\homeworkProblemName}{}%
\newcounter{homeworkProblemCounter}%
\newenvironment{homeworkProblem}[1][Problem \arabic{homeworkProblemCounter}]%
  {\stepcounter{homeworkProblemCounter}%
   \renewcommand{\homeworkProblemName}{#1}%
   \section{\homeworkProblemName}%
   \enterProblemHeader{\homeworkProblemName}}%
  {\exitProblemHeader{\homeworkProblemName}}%

\newcommand{\problemAnswer}[1]
  {\noindent\fbox{\begin{minipage}[c]{\columnwidth}#1\end{minipage}}}%

\newcommand{\problemLAnswer}[1]
  {\labelAnswer{\homeworkProblemName}{#1}}

\newcommand{\homeworkSectionName}{}%
\newlength{\homeworkSectionLabelLength}{}%
\newenvironment{homeworkSection}[1]%
  {% We put this space here to make sure we're not connected to the above.
   % Otherwise the changetext can do funny things to the other margin

   \renewcommand{\homeworkSectionName}{#1}%
   \settowidth{\homeworkSectionLabelLength}{\homeworkSectionName}%
   \addtolength{\homeworkSectionLabelLength}{0.25in}%
   \changetext{}{-\homeworkSectionLabelLength}{}{}{}%
   \subsection{\homeworkSectionName}%
   \enterProblemHeader{\homeworkProblemName\ [\homeworkSectionName]}}%
  {\enterProblemHeader{\homeworkProblemName}%

   % We put the blank space above in order to make sure this margin
   % change doesn't happen too soon (otherwise \sectionAnswer's can
   % get ugly about their \marginpar placement.
   \changetext{}{+\homeworkSectionLabelLength}{}{}{}}%

\newcommand{\sectionAnswer}[1]
  {% We put this space here to make sure we're disconnected from the previous
   % passage

   \noindent\fbox{\begin{minipage}[c]{\columnwidth}#1\end{minipage}}%
   \enterProblemHeader{\homeworkProblemName}\exitProblemHeader{\homeworkProblemName}%
   \marginpar{\fbox{\homeworkSectionName}}%

   % We put the blank space above in order to make sure this
   % \marginpar gets correctly placed.
   }%

%%%%%%%%%%%%%%%%%%%%%%%%%%%%%%%%%%%%%%%%%%%%%%%%%%%%%%%%%%%%%


%%%%%%%%%%%%%%%%%%%%%%%%%%%%%%%%%%%%%%%%%%%%%%%%%%%%%%%%%%%%%
% Make title
\title{\vspace{2in}\textmd{\textbf{\hmwkClass:\ \hmwkTitle}}\\\normalsize\vspace{0.1in}\small{Due\ on\ \hmwkDueDate}\\\vspace{0.1in}\large{}\vspace{3in}}
\date{}
%\author{\textbf{\hmwkAuthorName}}
%%%%%%%%%%%%%%%%%%%%%%%%%%%%%%%%%%%%%%%%%%%%%%%%%%%%%%%%%%%%%

\def\Figure#1#2#3#4#5{
\begin{figure}[htp]
%\epsscale{#4}
\includegraphics[scale=#4,angle=#5,width=7in]{#1}
\caption{#2}
\label{#3}
\end{figure}
}


\begin{document}
\begin{spacing}{0.5}
%\maketitle
\newpage



\section{Git Setup Revisited}

This is a walk-through of how to make a repository on github and share
it with Cameron and Adam.

\subsection{github.com}
Have you completed the other git tutorials?  If so, you should see
a window that looks like this one:

\Figure{Frame1_yourdir.png}
{What you should see at \url{http://github.com} if you completed
the previous tutorials.  The arrow points to your directory;
if it exists click it and move on to Section \ref{sec:setup}, otherwise
go to Section \ref{sec:makenew}}
{fig:yourdir}{0.5}{0}

\clearpage
\subsection{Make a New Repo}
\label{sec:makenew}
If you haven't completed the previous git tutorials, do so now.
This document provides some guidance.

\Figure{MakeNewRepo.png}
{Click the button in the top right to create a new repository.}
{fig:makenew}{0.5}{0}

\Figure{CreateRepository.png}
{Create your repository. \emph{Make sure} you select the ``private'' button.
If it asks you for money at this point, go to \url{http://github.com/edu}
and click ``I'm a student'' and fill in the form.  Then, return here.}
{fig:create}{0.5}{0}

\clearpage
\subsection{Set up your repository}
\label{sec:setup}
Set up your repository locally.

Before you do that, switch editors to \verb|nano|.  We were using gedit
as the editor, but I found out recently that that can lead to problems.\\
\verb|git config --global core.editor nano|

Do the following:
\begin{enumerate}
    \item Change directories to your assignments directory.
        You can probably do this with:\\
        \verb|cd ~/ASTR2600|
    \item \verb|git init|
    \item \verb|git add *|
    \item \verb|git commit -m "First Commit"|
\end{enumerate}

Go back to your repository on github.
You should see a screen like Figure \ref{fig:setup}.  Follow the highlighted
instructions, i.e. do the \texttt{git remote add} and \texttt{git push}
commands as shown.

\Figure{GitSetup.png}
{The setup instructions once you've created a repository and opened it, but
before you've added content.  Make sure the SSH button is selected.  You will
use the \texttt{git remote add} and \texttt{git push} commands shown highlighted
in the red box (don't try to just do \texttt{git remote add}, that is not a
complete command).}
{fig:setup}{0.5}{0}


\clearpage
\section{Add Cameron and Adam as collaborators}
\label{sec:collab}
This part is necessary for us to see the work you turn in.

In your repository, click the admin button (Figure \ref{fig:adminbutton})

\Figure{AdminButton.png}
{The admin button}
{fig:adminbutton}{0.5}{0}

Once in admin, click the Collaborators button (Figure \ref{fig:collabbutton})

\Figure{CollaboratorsButton.png}
{The Collaborators button}
{fig:collabbutton}{0.5}{0}

Then, add us as collaborators.  The usernames are:\\
keflavich\\
cwedge90\\

\Figure{AddCollaborators.png}
{The Add Collaborators dialog}
{fig:addcollab}{0.5}{0}

\end{spacing}
\end{document}

